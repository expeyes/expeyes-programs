\documentclass[12pt,a4paper]{report}
\usepackage{graphicx}
\usepackage[utf8]{inputenc}
\DeclareUnicodeCharacter{00A0}{\nobreakspace}
\usepackage[T1]{fontenc}
\usepackage[french]{babel}
\usepackage{times}
\begin{document}
\thispagestyle{empty}

Prefacio

El proyecto PHOENIX (\textsc{Physics with Home-made Equipment \& Innovative
Experiments}~: Physique avec un matériel «~maison~» \& des expériences
innovantes) comenzó en 2004 en \textsc{Inter-University Accelerator
Centre} con el objetivo de mejorar la educación científica en las 
universidades indias El desarrollo de equipos de laboratorio 
de bajo costo y la capacitación de docentes son las dos actividades 
principales de este proyecto..

\textsc{expEYES-Junior} es una versión avanzada de \textsc{expEYES} publicada
antes.Está diseñado para ser una herramienta de aprendizaje de experimentaión, 
válida para clases de secundaria y superiores. Hemos tratado de optimizar el 
diseño para hacerlo simple, flexible, robusto y económico.
El bajo precio lo hace accesible a las personas y esperamos ver a los estudiantes 
haciendo experimentos fuera de las cuatro paredes del laboratorio, que se cierra 
cuando suena la campana.

Este software se publicó bajo licencias \textsc{GNU General Public
License} et \textsc{CERN Open Hardware Licence}. El proyecto ha avanzado 
gracias a la participación activa y las contribuciones de la comunidad de usuarios 
y muchas otras personas externas de \textsc{IUAC}.
Agradecemos al Dr. D Kanjilal por las etapas necesarios para desarrollar 
esta nueva versión del trabajo de su desarrollador, Jithin B P, de CSpark Research.

El manual de usuario de \textsc{expEYES-Junior} se distribuye bajo 
licencia \textsc{GNU Free Documentation}.

Ajith Kumar B.P. ~~~~~~~~~(ajith@iuac.res.in) ~~http://expeyes.in

V V V Satyanarayana
\end{document}






