\documentclass[12pt,a4paper]{report}
\usepackage{graphicx}
\usepackage[utf8]{inputenc}
\DeclareUnicodeCharacter{00A0}{\nobreakspace}
\usepackage[T1]{fontenc}
\usepackage[french]{babel}
\usepackage{times}
\begin{document}
\thispagestyle{empty}

Préface

Le projet PHOENIX (\textsc{Physics with Home-made Equipment \& Innovative
Experiments}~: Physique avec un matériel «~maison~» \& des expériences
innovantes) a démarré en 2004 au \textsc{Inter-University Accelerator
Centre} avec l'objectif d'améliorer l'enseignement des sciences dans
les Universités Indiennes. Le développement de matériel de laboratoire
à bas coût et la formation des enseignants sont les deux activités
principales de ce projet.

\textsc{expEYES-17} est une version avancée du \textsc{expEYES} publié
plus tôt. Il est conçu pour être un outil d'apprentissage par l'exploration,
valide pour les classes de lycée et au-dessus. Nous avons essayé d'optimiser
la conception pour la rendre simple, flexible, robuste et bon marché.
Le prix bas le rend accessible aux individus et nous espérons voir
des étudiants réaliser des expériences en dehors des quatre murs du
laboratoire, qui ferme à la sonnerie de la cloche.

Ce logiciel est publié sous les licences \textsc{GNU General Public
License} et \textsc{CERN Open Hardware Licence}. Le projet a avancé
grâce aux participations actives et contributions de la communauté
des utilisateurs et de plusieurs autres personnes en dehors de l'\textsc{IUAC}.
Nous remercions le Dr D Kanjilal pour les étapes nécessaires à l'élaboration
de cette nouvelle version à partir du travail de son développeur,
Jithin B P, de CSpark Research.

Le manuel utilisateur de \textsc{expEYES-17} est distribué sous la
licence \textsc{GNU Free Documentation}.

Ajith Kumar B.P. ~~~~~~~~~(ajith@iuac.res.in) ~~http://expeyes.in

V V V Satyanarayana
\end{document}
